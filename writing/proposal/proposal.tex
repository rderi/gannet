%%
%% This is file `main.tex' based on `sample-sigconf.tex' (q.v. for spurce of that,
%%
%% IMPORTANT NOTICE:
%% 
%% For the copyright see the original source file `sample-sigconf.tex'
%% in the `Sample' folder.
%%
%% For distribution of the original source see the terms
%% for copying and modification in the file samples.dtx.
%% 
%% This generated file may be distributed as long as the
%% original source files, as listed above, are part of the
%% same distribution. (The sources need not necessarily be
%% in the same archive or directory.)
%%
%% Commands for TeXCount
%TC:macro \cite [option:text,text]
%TC:macro \citep [option:text,text]
%TC:macro \citet [option:text,text]
%TC:envir table 0 1
%TC:envir table* 0 1
%TC:envir tabular [ignore] word
%TC:envir displaymath 0 word
%TC:envir math 0 word
%TC:envir comment 0 0
%%
%%
%% The first command in your LaTeX source must be the \documentclass command.

%% NOTE that a single column version is required for 
%% submission and peer review. This can be done by changing
%% the \doucmentclass[...]{acmart} in this template to 
%%\documentclass[manuscript,screen,review]{acmart}
%% This version is used for drafting and final submission
\documentclass[sigconf]{acmart}



%% 
%% To ensure 100% compatibility, please check the white list of
%% approved LaTeX packages to be used with the Master Article Template at
%% https://www.acm.org/publications/taps/whitelist-of-latex-packages 
%% before creating your document. The white list page provides 
%% information on how to submit additional LaTeX packages for 
%% review and adoption.
%% Fonts used in the template cannot be substituted; margin 
%% adjustments are not allowed.

%%
%% \BibTeX command to typeset BibTeX logo in the docs
\AtBeginDocument{%
  \providecommand\BibTeX{{%
    \normalfont B\kern-0.5em{\scshape i\kern-0.25em b}\kern-0.8em\TeX}}}

    
%% Rights management information.  This information is sent to you
%% when you complete the rights form.  These commands have SAMPLE
%% values in them; it is your responsibility as an author to replace
%% the commands and values with those provided to you when you
%% complete the rights form.
\setcopyright{acmlicensed}
\copyrightyear{2024}
\acmYear{2024}
\acmDOI{XXXXXXX.XXXXXXX}
\iffalse
%% These commands are for a PROCEEDINGS abstract or paper.
\acmConference[Conference acronym 'XX]{Make sure to enter the correct
  conference title from your rights confirmation email}{June 03--05,
  2018}{Woodstock, NY}
%
%  Uncomment \acmBooktitle if th title of the proceedings is different
%  from ``Proceedings of ...''!
%
%\acmBooktitle{Woodstock '18: ACM Symposium on Neural Gaze Detection,
%  June 03--05, 2018, Woodstock, NY} 
\acmISBN{978-1-4503-XXXX-X/18/06}
\fi

%%
%% Submission ID.
%% Use this when submitting an article to a sponsored event. You'll
%% receive a unique submission ID from the organizers
%% of the event, and this ID should be used as the parameter to this command.
%%\acmSubmissionID{123-A56-BU3}

%%
%% For managing citations, it is recommended to use bibliography
%% files in BibTeX format.
%%
%% You can then either use BibTeX with the ACM-Reference-Format style,
%% or BibLaTeX with the acmnumeric or acmauthoryear sytles, that include
%% support for advanced citation of software artefact from the
%% biblatex-software package, also separately available on CTAN.
%%
%% Look at the sample-*-biblatex.tex files for templates showcasing
%% the biblatex styles.
%%

%%
%% The majority of ACM publications use numbered citations and
%% references.  The command \citestyle{authoryear} switches to the
%% "author year" style.
%%
%% If you are preparing content for an event
%% sponsored by ACM SIGGRAPH, you must use the "author year" style of
%% citations and references.
%% Uncommenting
%% the next command will enable that style.
%%\citestyle{acmauthoryear}

%%
%% end of the preamble, start of the body of the document source.

%% % Location of your graphics files for figures, here a sub-folder to the main project folder
\graphicspath{{./images/}} 


\begin{document}



%%
%% The "title" command has an optional parameter,
%% allowing the author to define a "short title" to be used in page headers.
\title{Optimizing Neural Network Hardware Latency and Energy Use with Genetic Algorithms}

%%
%% The "author" command and its associated commands are used to define
%% the authors and their affiliations.
%% Of note is the shared affiliation of the first two authors, and the
%% "authornote" and "authornotemark" commands
%% used to denote shared contribution to the research.
\author{Regina Deri}
\email{rderi25@amherst.edu}
\affiliation{%
  \institution{Amherst College}
  \city{Amherst}
  \state{Massachusetts}
  \country{USA}
  \postcode{01002}
}



%%
%% By default, the full list of authors will be used in the page
%% headers. Often, this list is too long, and will overlap
%% other information printed in the page headers. This command allows
%% the author to define a more concise list
%% of authors' names for this purpose.
\renewcommand{\shortauthors}{Regina Deri}

%%
%% The abstract is a short summary of the work to be presented in the
%% article.
%\begin{abstract}
  
%\end{abstract}

%%
%% The code below is generated by the tool at http://dl.acm.org/ccs.cfm.
%% Please copy and paste the code instead of the example below.
%%
\iffalse
\begin{CCSXML}
<ccs2012>
 <concept>
  <concept_id>00000000.0000000.0000000</concept_id>
  <concept_desc>Do Not Use This Code, Generate the Correct Terms for Your Paper</concept_desc>
  <concept_significance>500</concept_significance>
 </concept>
 <concept>
  <concept_id>00000000.00000000.00000000</concept_id>
  <concept_desc>Do Not Use This Code, Generate the Correct Terms for Your Paper</concept_desc>
  <concept_significance>300</concept_significance>
 </concept>
 <concept>
  <concept_id>00000000.00000000.00000000</concept_id>
  <concept_desc>Do Not Use This Code, Generate the Correct Terms for Your Paper</concept_desc>
  <concept_significance>100</concept_significance>
 </concept>
 <concept>
  <concept_id>00000000.00000000.00000000</concept_id>
  <concept_desc>Do Not Use This Code, Generate the Correct Terms for Your Paper</concept_desc>
  <concept_significance>100</concept_significance>
 </concept>
</ccs2012>
\end{CCSXML}

\ccsdesc[500]{Do Not Use This Code~Generate the Correct Terms for Your Paper}
\ccsdesc[300]{Do Not Use This Code~Generate the Correct Terms for Your Paper}
\ccsdesc{Do Not Use This Code~Generate the Correct Terms for Your Paper}
\ccsdesc[100]{Do Not Use This Code~Generate the Correct Terms for Your Paper}

%%
%% Keywords. The author(s) should pick words that accurately describe
%% the work being presented. Separate the keywords with commas.
\keywords{Do, Not, Us, This, Code, Put, the, Correct, Terms, for,
  Your, Paper}

%% The following are not a requirement, delete if not using
\received{20 February 2024}  %% inital submission date
\received[revised]{12 March 2024} %% interim new draft
\received[accepted]{5 June 2024}  %% publication version
\fi
%%
%% This command processes the author and affiliation and title
%% information and builds the first part of the formatted document.
\maketitle

\section{Introduction \& Motivation}

Genetic algorithms can be great tools in finding and optimizing solutions to complex problems, 
such as those involving design. Previously, they have been successfully used to evolve circuits, 
antennas, or even hardware modules for quantum computing (for some examples, see Koza et al., 2005). 

Neural network design space exploration also seems to be an area where genetic programming could be fruitfully applied. 
When developing neural networks, it's often difficult to consider the trade-offs between latency, energy use, and accuracy.
In this project, I will create a genetic algorithm framework that will optimize the hardware latency and energy use of neural networks. 
In addition, I aim to analyze the potentials of GA based neural network optimizations by experimenting with both completely new
and prior models.  

\section{Background \& Related Work}

\subsection{Key definitions}
\begin{itemize}
\setlength\itemsep{1em}
\item[] \emph{Neural networks}: Neural networks is one of the most commonly used machine learning models, inspired by the functioning of the human brain. 
                                They consist of layers of interconnected neurons that process and transform data.
                                Neural networks have many applications, such as image recognition, natural language processing, and autonomous vehicles.

\item[] \emph{Genetic algorithms (GAs)}: 
                                Genetic algorithms are a type of optimization algorithm inspired by the process of natural selection. 
                                They work by evolving a population of steadily improving solutions over multiple generations, 
                                using an array of operators such as mutation and crossover.
                                Genetic algorithms have been successfully used in a variety of domains, such as circuit design, antenna design, and robotics.

\item[] \emph{Design Space Exploration (DSE)}
                                Design space exploration is the process of exploring the space of possible designs for a given problem, in order to find the best design
                                according to some metric. Since the process is automatized, it often has the power to reveal insights into design that otherwise would not have been found.
\setlength\itemsep{0em}
\end{itemize}
\subsection{Related work}
\begin{itemize}
  \setlength\itemsep{1em}
  \item[] \emph{\href{https://www.sciencedirect.com/science/article/pii/S1877050921012801?ref=pdf_download&fr=RR-2&rr=868a43d10db70f71}{Domashova et al. 2021}}: Neural networks is one of the most commonly used machine learning models, inspired by the functioning of the human brain. 
                                  The paper attempts to find optimal architecture for neural networks using genetic algorithms. While the paper doesn't focus on hardware latency and energy use, it provides a good case study for this project.
  
  \item[] \emph{\href{https://ieeexplore.ieee.org/document/9360462}{Mei et al. 2022}}: 
                                  This paper introduces ZigZag, a tool that will be used as an essential part of this project to measure the hardware latency and energy use of candidate solutions.
  
  \item[] \emph{\href{https://www.researchgate.net/publication/321021361_Neural_Networks_Optimization_through_Genetic_Algorithm_Searches_A_Review}{Chiroma et al. 2017}}:
                                  This paper is a review of similar attempts at optimization neural networks with genetic algorithms. It gives a thorough overview of the field, and the referenced papers could prove very useful for my work.

  \item[] \emph{\href{https://link.springer.com/book/10.1007/978-3-662-44874-8}{Eiben et al. 2015}}:
                                  This book is essentialy a textbook on genetic algorithms. It will be useful when I get to the point of fine-tuning aspects of my genetic algorithm framework, such as survivor selection functions.  
  
\end{itemize}

\section{Methods \& Tools}

The project is implementation-heavy, and will involve the creation of a genetic algorithm system 
that optimizes PyTorch sequential models for hardware latency and energy use. Subsequently, the system will
be evaluated on Amherst College's High Performance Cluster (HPC), on a diverse selection of existing models and neural networks
generated from scratch by the genetic algorithm framework itself.

One limitation of my methods is that the measured fitness scores of models will be specific to the hardware/accelerator designs used. I am still looking for a way
to generalize these results, possibly by using multiple accelerator designs.

Since much of the project involves implementation, there's only a small number of tools that I will use:
\begin{itemize}
    \item \emph{Python}: The majority of the project's code will be implemented in Python.
    \item \emph{PyTorch}: PyTorch is a popular open-source machine learning library for Python. 
                   It was created by Meta's AI Research lab and is now governed by the PyTorch foundation.
                   It's one of the most important tools for this project, and will be the tool of choice in creating, training, and evaluating neural networks.
    \item \emph{ZigZag}: ZigZag is a tool that helps estimate the hardware latency and energy use of neural networks running on a given accelrator design.
                  It also provides a heuristic mapper to map neural networks to accelerator designs, however, I will mainly use it for the former purpose.
                  Zigzag was first proposed in a paper by Mei et al. (2021), and has since been a subject of nine other papers. It requires an AI model and an accelerator
                  design as input, and is able to output the estimated latency and energy use of the model running on the accelerator. It will be called from the genetic algorithm framework
                  to evaluate the fitness score of the neural networks.
                   
\end{itemize}

\section{Preliminary Results or Analysis}
So far, I have implemented most of my genetic algorithm framework, including genomes, fitness measurement, crossover, mutation, and selection.
There is still some work left to connect these modules, after which the framework will be ready for testing.

\section{Outline of Final Paper}
\begin{enumerate}
    \item Introduction
    \item Background and Related Work
    \item Description of the genetic algorithm framework
    \item Case study 1: optimizing model created from scratch
    \item Case study 2: optimizing existing model
    \item Discussion
    \item Conclusion
\end{enumerate}

\section{Up-to-date Timeline for Completion}

\begin{enumerate}
  \item \textbf{Week of March 24}:
  \begin{itemize}
    \item Finish GA framework
    \item Begin comprehensive code testing
  \end{itemize}
  \item \textbf{Week of March 31}:
  \begin{itemize}
    \item Plug in to ZigZag
    \item Prepare HPC scripts, run preliminary test
    \item Find a way to organize the massive amount of generated results
  \end{itemize}
  \item \textbf{Week of April 7}:
  \begin{itemize}
    \item Optimize if needed
    \item Begin the measurements that will be in the final paper
    \item Get started with writing the paper
  \end{itemize}
  \item \textbf{Week of April 14}:
  \begin{itemize}
    \item Start analyzing results
    \item The time to redo measurements if something goes wrong
  \end{itemize}
  \item \textbf{Week of April 21}:
  \begin{itemize}
    \item Potentially begin experimenting with different GA methods (e.g., in term of survivor selection)
    \item Continue writing paper
  \end{itemize}
  \item \textbf{Week of April 28}:
  \begin{itemize}
    \item Most of the code and results should be ready by this week
    \item The week will be spent finalizing the first draft
  \end{itemize}
  \item \textbf{Week of May 5}:
  \begin{itemize}
    \item Finish first draft
  \end{itemize}
\end{enumerate}




%%
%% The next two lines define the bibliography style to be used, and
%% the bibliography file.
%\bibliographystyle{ACM-Reference-Format}
%\bibliography{references.bib}
%%
\end{document}
\endinput
%%
%% End of file `main.tex'.